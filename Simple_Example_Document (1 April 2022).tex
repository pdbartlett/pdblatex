\documentclass[11pt,a4paper]{article}
\title{Simple Example Document}
\author{Sophie Bartlett}
\date{1 April 2022}
\usepackage[section,nottoc,notbib]{tocbibind}
\usepackage{listings}
\usepackage[style=authoryear-ibid,backend=biber]{biblatex}
\usepackage{makeidx}
\usepackage{csquotes}
\usepackage{fancyhdr}
\usepackage{amsmath}

% Make title and author available
\makeatletter
\let\inserttitle\@title
\let\insertauthor\@author
\makeatother
% Put title and author in header
\pagestyle{fancy}
\fancyhf{}
\lhead{``\inserttitle''}
\rhead{\insertauthor}
\cfoot{\thepage}
\addtolength{\headheight}{2pt} % space for the rule
\makeindex
\addbibresource{Simple_Example_Document (1 April 2022).bib}
\begin{document}
\maketitle

\tableofcontents


\section{Introduction}

This document is intended to document all the things that can be done with
``non-weird'', Markdown-inspired syntax.

\section{Features}

\subsection{Basic}

These are some very basic features that can be used:
\begin{enumerate}
\item 
title and date taken from name of \verb|.md| one, or uses today's date if none specified

\item 
author defaults to Sophie Bartlett

\item 
\verb|.bib| file with same name as \verb|.md| loaded automatically

\item 
auto-conversion of double quotes (see \textit{Introduction} above)

\item 
\textbf{bold}, \textit{italic} and \verb|fixed-width| text

\item 
numbered lists (like this one)

\item 
table of contents, lists of figures/tables, bibliography and index placement

\item 
parenthetic citations \parencite{Jones2000} as well as direct ones, like \textcite{Smith2001}

\item 
simple indexing\index{indexing} of terms\index{terms} using hopefully intuitive syntax\index{syntax}

\item 
inline mathematical expressions, like $y = f(x)$

\item 
sections and subsections

\end{enumerate}

\subsection{Additional}

Here are a few more features, broken out separately mainly to demonstrate
subsections, but it also gives a chance to show blockquotes:
\begin{displayquote}

Blockquotes, like this one, can be used when  you need to add a sizeable
portion of text from a book, speech or other source, and format it so that it
stands out more clearly.
\end{displayquote}

It also provides an opportunity to see how subsequent paragraphs are formatted
(they are indented rather than ``skipping space'', though this can be changed at
a later date if it's preferable.

It also goves a chances to show a different type of list. This one is
bulleted---also known as unordered---as opposed to the numbered/ordered list used
in the previous section.
\begin{itemize}
\item 
bulleted lists (like this one)

\item 
the different dashes: hyphens (e.g. side-effect), \verb|en| dashes (e.g. 1--2 ideas),
or \verb|em| dashes as shown in the ``parenthetical'' example above

\item 
addition of appendices (lettered instead of numbered as sections are)

\item 
manual page breaks

\end{itemize}
\pagebreak

\appendix

\section{Extra information}

Lorem ipsum dolor sit amet, consectetur adipiscing elit. Pellentesque vitae diam
nec libero ultrices placerat id et tellus. Proin vestibulum, sapien eget lobortis
fermentum, nisi leo malesuada nibh, tempor tincidunt leo nibh in elit. Mauris
fermentum nunc sed pellentesque eleifend. Nunc eget libero eu elit placerat
feugiat nec lacinia elit. Aenean ultrices lobortis ex eget gravida. Suspendisse
in erat viverra, aliquam nulla a, gravida diam. Vivamus at viverra mauris. Mauris
vitae mauris arcu. Proin eu erat odio.
\pagebreak

\addcontentsline{toc}{section}{\bibname}
\printbibliography


\printindex

\end{document}
